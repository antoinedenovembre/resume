% ======================================================================
% ======================= Parcours professionnel =======================
% ======================================================================

    \section{Parcours professionnel}
        \begin{twocolentry}{
			\textbf{Brignais, France} \\
			\textit{Sept 2022 - Août 2025}
            }{
            \textbf{Safety Tech} \\
            \textit{Alternant Ingénieur en Traitement d'Image}
            }
        \end{twocolentry}

        \begin{onecolentry}
            \begin{highlights}
                \item Intégration de logiciels cross-compilés dans une image Linux \textbf{Buildroot} et un \textbf{RTOS} pour une cible embarquée grâce à \textbf{Docker} et \textbf{CMake}.
                \item Développement logiciel en C/C++ par l'utilisation de \textit{design patterns} et de scripts \textbf{Bash} pour l'automatisation de tâches, ainsi que la maintenance de dispositifs embarqués (accès série/SSH, \textbf{flashage}).
                \item Conception et maintenance de pipelines \textbf{GitLab CI/CD} pour automatiser la compilation, les tests et le déploiement.
                \item Travail avec de la communication \textbf{CAN}, \textbf{Ethernet} et \textbf{MQTT}, connaissance du \textbf{Cycle en V} en contexte industriel.
            \end{highlights}
        \end{onecolentry}

        \begin{twocolentry}{
			\textbf{Gjøvik, Norvège} \\
			\textit{Août 2024 - Oct 2024}
            }{
			\textbf{NTNU Gjøvik} \\
			\textit{Stage en traitement d'image par apprentissage profond}
            }
        \end{twocolentry}

        \begin{onecolentry}
            \begin{highlights}
                \item Entraînement et évaluation de modèles \textbf{EfficientDet} et \textbf{YOLOv8} pour la détection d'animaux dans des enclos via \textbf{Ultralytics} et \textbf{TensorFlow}
                \item Mise en œuvre d'un algorithme d'optimisation \textbf{Barlow Twins} pour l'apprentissage non-supervisé sur le modèle EfficientDet
            \end{highlights}
        \end{onecolentry}

% ======================================================================
% ======================== Parcours académique =========================
% ======================================================================
    
    \section{Parcours académique}
        \begin{twocolentry}{
            \textbf{Saint-Étienne, France}\\
            \textit{Sept 2022 - Aujourd'hui}
            }{
            \textbf{Télécom Saint-Étienne}\\
            \textit{Diplôme d'Ingénieur en Image, Photonique et Smart-Industries}
            }
        \end{twocolentry}


        \begin{onecolentry}
            \begin{highlights}
                \item \textbf{Cours pertinents :} Deep Learning (en Python), Traitement d'Image, Imagerie 3D
                \item \textbf{Informations pertinentes :} Alternance avec Safety Tech à Brignais, France
            \end{highlights}
        \end{onecolentry}

		\begin{twocolentry}{
			\textbf{Chicoutimi, Canada} \\
			\textit{Jan 2022 - Mai 2022}
            }{
            \textbf{Université du Québec à Chicoutimi} \\
            \textit{Semestre d'échange de DUT Informatique}
            }
        \end{twocolentry}

        \begin{onecolentry}
            \begin{highlights}
                \item \textbf{Cours pertinents :} Calcul avancé II, Forage de données (en Python)
            \end{highlights}
        \end{onecolentry}

		\begin{twocolentry}{
			\textbf{Clermont-Ferrand, France} \\
			\textit{Sept 2020 - Juil 2022}
            }{
            \textbf{Université Clermont Auvergne} \\
            \textit{DUT en Informatique}
            }
        \end{twocolentry}

        \begin{onecolentry}
            \begin{highlights}
                \item \textbf{Cours pertinents :} Programmation Orientée Objet, C++, Algorithmique avancé
            \end{highlights}
        \end{onecolentry}

% ======================================================================
% ======================= Compétences et intérêts ======================
% ======================================================================

    \section{Compétences et intérêts}
        \begin{onecolentry}
            \textbf{Langues :} Français (natif), Anglais (C1, TOEIC 990/990) 
        \end{onecolentry}

        \begin{onecolentry}
            \textbf{Compétences techniques :} C, C++, Réseaux, Machine et Deep Learning en Python (Ultralytics, TensorFlow), Programmation web, Traitement d'image, Gestion de projet SCRUM
        \end{onecolentry}

        \begin{onecolentry}
            \textbf{Intérêts :} Tendances technologiques, Jeux-vidéos, Voyage, Cinéma
        \end{onecolentry}